% TODO: ПОМЕНЯЙТЕ НА СВОЙ ПУТЬ ПРИ СБОРКЕ
\renewcommand{\mainFolder}{/home/aleksei/github_projects/latex_works/informatics}
\renewcommand{\myFolder}{\mainFolder/lecture_\arabic{lectureNo}/}

%	LECTURE INFO
%----------------------------------------------------------------------------------------
\renewcommand{\lectureSubject}{
	Двоичное дерево поиска. Продолжение.	
}
%----------------------------------------------------------------------------------------
%
% Главное - указывать в путях для добавляемых файлов переменную \myFolder.
% Так, новый cpp файл должен иметь путь \cppfile{\myFolder/main}{....} (указывается без расширения)
% Нельзя писать \cppfile{main}{...} !!!
% Если не выполнить это требование, большой проект не будет собираться


\begin{lecture}[\lectureSubject]
	
	\begin{lecSection}[Класс Дерево. Продолжение.]
		\pythonfile{\myFolder/Tree}{Класс Дерево}{tree_class}
	\end{lecSection}
	\begin{lecSection}[Балансировка дерева]
		Двоичное дерево поиска \textsf{сбалансированно}, если для каждой его вершины высота левого и правого поддерева отличаются не более чем на единицу.
		
		%нарисуем закошеное дерево 
		\textbf{Инвариант}: та вершина, которая левее других вершин, должна остаться левее всех вершин, т.е двигать вверх--вниз вершины можно, но влево--вправо двигать нельзя, иначе нарушится последовательность чисел.
		
		Алгоритм балансировки подробно описан в \href{https://ru.wikipedia.org/wiki/%D0%90%D0%92%D0%9B-%D0%B4%D0%B5%D1%80%D0%B5%D0%B2%D0%BE#.D0.91.D0.B0.D0.BB.D0.B0.D0.BD.D1.81.D0.B8.D1.80.D0.BE.D0.B2.D0.BA.D0.B0}{википедии}.
			
			\textsf{АВЛ--дерево} --- сбалансированное по высоте двоичное дерево поиска: для каждой его вершины высота её двух поддеревьев различается не более чем на 1.
			
			
			\textsf{Красно--чёрное дерево} --- это одно из самобалансирующихся двоичных деревьев поиска, гарантирующих логарифмический рост высоты дерева от числа узлов и быстро выполняющее основные операции дерева поиска: добавление, удаление и поиск узла. Сбалансированность достигается за счёт введения дополнительного атрибута узла дерева --- «цвета». Этот атрибут может принимать одно из двух возможных значений --- «чёрный» или «красный».
			
			\textbf{Свойства} красно--черного дерева:
			\begin{enumerate}
				\item Узел либо красный, либо чёрный.
				\item Корень --- чёрный. (В других определениях это правило иногда опускается. Это правило слабо влияет на анализ, так как корень всегда может быть изменен с красного на чёрный, но не обязательно наоборот).
				\item Все листья --- чёрные.
				\item Оба потомка каждого красного узла --- чёрные.
				\item Всякий простой путь от данного узла до любого листового узла, являющегося его потомком, содержит одинаковое число чёрных узлов.
			\end{enumerate}
	\end{lecSection}
\end{lecture}
\stepcounter{lectureNo}