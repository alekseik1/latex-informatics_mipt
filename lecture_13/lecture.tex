% TODO: ПОМЕНЯЙТЕ НА СВОЙ ПУТЬ ПРИ СБОРКЕ
\renewcommand{\mainFolder}{/home/aleksei/github_projects/latex_works/informatics}
\renewcommand{\myFolder}{\mainFolder/lecture_\arabic{lectureNo}/}

%	LECTURE INFO
%----------------------------------------------------------------------------------------
\renewcommand{\lectureSubject}{Исключения в С++}
%----------------------------------------------------------------------------------------
%
% Главное - указывать в путях для добавляемых файлов переменную \myFolder.
% Так, новый cpp файл должен иметь путь \cppfile{\myFolder/main}{....} (указывается без расширения)
% Нельзя писать \cppfile{main}{...} !!!
% Если не выполнить это требование, большой проект не будет собираться



\begin{lecture}[\lectureSubject]
	\begin{lecSection}[Понятие исключения]
		Пусть некая функция \textbf{main()} вызывает функции \textbf{A,B,C,D} по схеме $A\rightarrow B\rightarrow C\rightarrow D$. Что произойдет, если одна из них не сможет выполнить свои обязательства (встретится какая-нибудь ошибка, ресурсы будут недоступны и т.п.)?
		\begin{lecSubsection}{Интерфейс функции}
			Прежде чем ответить на этот вопрос, разберемся с таким понятием, как \textit{интерфейс функции}. Интерфейс функции -- это некий абстрактный набор, содержащий характеризующую функцию информацию. Он включает \textit{имя, смысл, типы параметров, их допустимые значения и \underline{возможные} исключения}. Интерфейс функции должен быть продуман перед разработкой, на этапе осмысливания архитектуры целого приложения.
		\end{lecSubsection}
	\end{lecSection}
	\begin{lecSection}[Исключения в С++]
		\label{exc1}
		В С++, как и во многих языках программирования, наша ошибка выполнения будет обработана следующим образом: неисправная функция создаст \textbf{исключение} и вернет вызвавшей ее функции это исключение. Та, с свою очередь, должна будет его обработать дальше. Исключения, порождаемые функцией, являются самостоятельными типами. Они имеют свои поля и атрибуты.
	\end{lecSection}
	\begin{lecSection}[Стек вызова]
			Когда одна функция вызывает другую, вторая отправляется в \textit{стек вызовов}. Это такая очередь, содержащая функции в том порядке, в каком они вызывались. В стеке лежат функции и \textit{адреса возврата} -- адрес того места функции в оперативной памяти, где была вызвана новая функция (чтобы впоследствии вернуться строго в то место, где выполнения программы прервались на вызов другой функции). Помимо адресов в стек вызовов уходят локальные переменные, место для аргументов функции, место для возвращаемого значения (или для адреса, если возвращаемое значение лежит в динамической памяти). При этом передаваемые аргументы будут передаваться по обычным правилам (с использованием конструктора копирования для объектов и т.п.).
	\end{lecSection}
	\begin{lecSection}[Порядок вылета из программы]
		Что произойдет, если вылетит исключение и не будет обработано? Программа вылетит, причем весь код дальше выполнен \underline{не} будет. Это может привести к утечке памяти (если мы таким способом пропустим \textbf{delete}, освобождающий память) и другим проблемам (о них ниже).
	\end{lecSection}
	\begin{lecSection}[Реализация исключений в С++]
		Теперь посмотрим, какова реализация механизма исключений в С++. Работа с опасным кодом в этом языке осуществляется следующим образом:
		\cppfile{\myFolder/1}{Синтаксис исключений}{1}
		При этом \textbf{ExceptionType} -- это тип исключения (см. \ref{exc1}). Если в опасный код выбросит исключение типа \textbf{ExceptionType}, то программа начнет выполнять код в блоке \textbf{catch}. Если это исключение другого типа, оно \underline{не} будет обработано.
	\end{lecSection}
	\begin{lecSection}[Проблемы исключений]
		В процессе 'вылета' из-за исключения остальные объекты нашей программы будут удалены. А для этого нужно вызвать их деструкторы. Что произойдет, если при этом деструктор сам выбросит исключение, не знает даже Т.Ф., поэтому деструкторы надо писать максимально простыми и \underline{не} выбрасывающими исключения.
		
		Еще одна проблема, которая может случиться -- исключение во время выполнения конструктора. При таком исходе объект не будет до конца создан и для него не будет вызван деструктор. Это тоже может вести к утечкам памяти.
	\end{lecSection}
	\begin{lecSection}[Иерархия исключений]
		Сами исключения, как объекты классов, имеют свою иерархию. Самое базовое исключение есть экземпляр класса \textbf{BaseException}. От него наследуются уже специфичные исключения, например, \textbf{ArithmeticException} -- исключение в арифметике (деление на ноль и т.п.). Все эти типы можно передать в \textbf{catch}, тем самым указав типа ошибки, который будет обработан в конкретном блоке.
		\begin{lecSubsection}{Default exception}
			Исключение вида:
			\cppfile{\myFolder/3}{Default exception}{3}
			является исключением по умолчанию. Это, по-простому говоря, исключение на все случаи жизни. Т.Ф. не рекомендует им пользоваться, ведь этот тип практически ничего не будет знать о деталях ошибки.
		\end{lecSubsection}
	\end{lecSection}
	\begin{lecSection}[Вторичные исключения]
		При обработке исключения в блоке \textbf{catch} может случиться такое, что с какими-то проблемами наш блок будет в состоянии разобраться, а с какими-то -- нет. В последнем случае \textbf{catch} может сам выкинуть exception и передать его уже выше. Это делается командой \textbf{throw}.
		\cppfile{\myFolder/2}{Пример вторичного исключения}{2}
	\end{lecSection}
\end{lecture}
\stepcounter{lectureNo}